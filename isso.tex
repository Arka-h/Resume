%%%%%%%%%%%%%%%%%
% This is an example CV created using altacv.cls (v1.1.5, 1 December 2018) written by
% LianTze Lim (liantze@gmail.com), based on the
% Cv created by BusinessInsider at http://www.businessinsider.my/a-sample-resume-for-marissa-mayer-2016-7/?r=US&IR=T
%
%% It may be distributed and/or modified under the
%% conditions of the LaTeX Project Public License, either version 1.3
%% of this license or (at your option) any later version.
%% The latest version of this license is in
%%    http://www.latex-project.org/lppl.txt
%% and version 1.3 or later is part of all distributions of LaTeX
%% version 2003/12/01 or later.
%%%%%%%%%%%%%%%%

%% If you are using \orcid or academicons
%% icons, make sure you have the academicons
%% option here, and compile with XeLaTeX
%% or LuaLaTeX.
% \documentclass[10pt,a4paper,academicons]{altacv}

%% Use the "normalphoto" option if you want a normal photo instead of cropped to a circle
% \documentclass[10pt,a4paper,normalphoto]{altacv}

\documentclass[10pt,a4paper,ragged2e]{altacv}

%% AltaCV uses the fontawesome and academicon fonts
%% and packages.
%% See texdoc.net/pkg/fontawecome and http://texdoc.net/pkg/academicons for full list of symbols. You MUST compile with XeLaTeX or LuaLaTeX if you want to use academicons.

% Change the page layout if you need to
\geometry{left=1cm,right=9cm,marginparwidth=6.8cm,marginparsep=1.2cm,top=1.25cm,bottom=1.25cm}

% Change the font if you want to, depending on whether
% you're using pdflatex or xelatex/lualatex
\ifxetexorluatex
  % If using xelatex or lualatex:
  \setmainfont{Carlito}
\else
  % If using pdflatex:
  \usepackage[utf8]{inputenc}
  \usepackage[T1]{fontenc}
  \usepackage[default]{lato}
\fi

% Change the colours if you want to
\definecolor{VividPurple}{HTML}{140E77}
\definecolor{SlateGrey}{HTML}{2E2E2E}
\definecolor{LightGrey}{HTML}{666666}
\colorlet{heading}{VividPurple}
\colorlet{accent}{VividPurple}
\colorlet{emphasis}{SlateGrey}
\colorlet{body}{LightGrey}

% Change the bullets for itemize and rating marker
% for \cvskill if you want to
\renewcommand{\itemmarker}{{\small\textbullet}}
\renewcommand{\ratingmarker}{\faCircle}

%% sample.bib contains your publications
\addbibresource{sample.bib}

\begin{document}
\name{ARKA HALDI}
\tagline{“Everything is theoretically impossible, until it is done. -Robert A. Heinlein”}
% Cropped to square from https://en.wikipedia.org/wiki/Marissa_Mayer#/media/File:Marissa_Mayer_May_2014_(cropped).jpg, CC-BY 2.0
\photo{2.5cm}{Arka_resume} % Arka_resume
\personalinfo{%
  % Not all of these are required!
  % You can add your own with \printinfo{symbol}{detail}
  \email{arkahaldi@iisc.ac.in}
    \phone{9497327717}
  \location{Mumbai, India}
  \linkedin{www.linkedin.com/in/arkahaldi}
  \github{https://github.com/Arka-h} 
  \kaggle{https://www.kaggle.com/arka4h}
  \hackerank{https://www.hackerrank.com/arka\_h}
  %If you want to use this field (and also other academicons symbols), add "academicons" option to \documentclass{altacv}%
}

%% Make the header extend all the way to the right, if you want.
\begin{fullwidth}
\makecvheader
\end{fullwidth}

%% Depending on your tastes, you may want to make fonts of itemize environments slightly smaller
\AtBeginEnvironment{itemize}{\small}

%% Provide the file name containing the sidebar contents as an optional parameter to \cvsection.
%% You can always just use \marginpar{...} if you do
%% not need to align the top of the contents to any
%% \cvsection title in the "main" bar.

% \cvsection[page1sidebar]{MOTIVATION}
% \begin{itemize}
% \item I like exploring new technologies, experimenting and problem solving. Having a keen interest in online competitions, and hackathons, I am motivated and always ready to learn more through workshops, coding competitions, hackakthons, conferences, mentors, follow forums, and journals on better coding techniques etc.
% \end{itemize}

\cvsection[page1sidebar]{Education}
\cvproject{M.Tech res (CDS)}{7.6 CGPA}{2025}{Indian Institute of Technology, Bangalore}
\cvproject{B.Tech (CSE)}{9.21 CGPA}{2022}{Sardar Patel Institute of Technology}
\cvproject{HSC (XII)}{93.3\%}{2018}{PACE Junior College, Dadar}
\cvproject{CBSE (X) }{96.6\%}{2016}{Navy Children School, Kochi}


\cvsection{Experience}
\cvprojlinkless{Barclays}{Aug 2022 - July 2023}{
\begin{itemize}
\item Worked as a Graduate Analyst, in Data-QE role. Daily responsibilities include creating detailed test cases, conducting reviews to ensure accuracy, and collaborating across teams to fix bugs.
% \item Delivered a critical project, taking end-to-end QE ownership; engaging with the product owner, developer, and AWS team despite frequent changes, and ensured timely delivery with quality coverage.
\item Recognized for Drive within organisation \& active participant in various programs, events, hackathons etc.
\item Proposed and created a POC for automation scripts, streamlining the process and reducing manual work by 70\%, saving 80\% time of testcase creation/management \& code checkouts.
\item Skills/tools - SQL, PL-SQL, Ab Initio GDE, ETL processes, Data Warehousing, Data Architecture, Operating System (UNIX coding), Apache Hadoop(HQL), Data Migration, Data Cleansing, Cyara Automation, IVR testing. 
\end{itemize}
}
\divider

% \cvprojlinkless{UniConn Internship}{Oct 2020 - Aug 2021}
% {
% \begin{itemize}
% \item Was part of the core tech team & recommended database architectural improvements, design solutions and integration solutions.
% \item Provided continued development, and also created & maintained custom React components. Main responsibilities included integration of frontend / backend.
% \item Tools Used - Nodejs, MongoDB Atlas, mongoose, Express, React, Passport, Redux, GridFS, OpenAPI3
% \end{itemize}
% }
\cvsection{Publications}
\cvproject{\href{https://doi.org/10.1007/978-3-031-31407-0_40}{CV Based Mechanism for Detecting Fire and Its Classes}}{ICCVIP, Springer}{May 2023}{}
\begin{itemize}
    \item Our work presents a fast, robust fire detection model that localises, and classifies the fire into it's "class" depending on the source of fire in real-time from source footage(CCTV). This is important to first responders that can use the appropriate methods to subdue the fire, according to it's type.
    \item Skills/tools: cv2, yolov5, pytorch, roboflow, dataset creation, data augmentation
\end{itemize}

\cvsection{UG Projects}
\cvproject{Job Recruitment App}{https://github.com/Arka-h/JobApp}{Nov 2020}{}
\begin{itemize}
\item UI/UX centric Job application portal for Job seekers and Employers, to manage and review candidates/applied jobs.
\item Lead a team of 3, we designed and built whole application within a week.
\item Tools Used - React, MongoDB Atlas, Express, HTML, CSS, GridFS
\end{itemize}
% \divider

% \cvproject{Email Feedback}{https://github.com/Arka-h/EmailFeedback}{Feb 2019}{}
% \begin{itemize}
% \item Simplify sending surveys to customers and collecting feedback. Single user can create multiple surveys and collect information responses to email.
% \item Tools Used - MongoDB, Express, React, Redux, Node.js, Stripe API, ReactForms, SendGrid
% \end{itemize}
% \divider

% \cvproject{ToDo Project(24hr dev. challenge)}{https://github.com/Arka-h/ToDo-App}{Oct 2020}{}
% \begin{itemize}
% \item Simple intuitive ToDo list with Google auth-sessions and completed/pending lists per user, submitted as a recruitment challenge for UniConn.
% \item Tools Used - MongoDB, Express, Reactjs, Nodejs, Passportjs
% \end{itemize}
% \divider



% \divider



%  \cvsection{EXPERIENCES}
%  \cvachievement{\faTrophy}{JPMC Code For Good }{
% Participated in this prestigious hackathon, and one among randomly assigned team of 7, helped develop a data visualization app for waste management NGO }
% \divider
% \divider
% \cvachievement{\faTrophy}{HackIt2.0}{ Participated in HackIt hackathon, and helped develop a Recruitment Assisting Platform at K.J.Somaiya college.}
% \divider




\clearpage

% \cvsection[page2sidebar]{Publications}
% 
% \nocite{*}
% 
% \printbibliography[heading=pubtype,title={\printinfo{\faBook}{Books}},type=book]
% 
% \divider
% 
% \printbibliography[heading=pubtype,title={\printinfo{\faFileTextO}{Journal Articles}}, type=article]
% 
% \divider
% 
% \printbibliography[heading=pubtype,title={\printinfo{\faGroup}{Conference Proceedings}},type=inproceedings]
% 
%% If the NEXT page doesn't start with a \cvsection but you'd
%% still like to add a sidebar, then use this command on THIS
%% page to add it. The optional argument lets you pull up the
%% sidebar a bit so that it looks aligned with the top of the
%% main column.
% \addnextpagesidebar[-1ex]{page3sidebar}


\end{document}
